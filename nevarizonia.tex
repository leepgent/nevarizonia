% $Id$
% Copyright 2005-2010 Lee Gent (lee@leegent.net)

\documentclass[b5paper,11pt,titlepage,draft]{book}
\usepackage{charter}
\usepackage[b5paper]{geometry}
%\usepackage[urw-garamond]{mathdesign}

\title{Nevarizonia: An American Road Trip}
\author{Lee Gent}
\date{May -- June, 2005}
\begin{document}
\maketitle
\tableofcontents
\frontmatter
\section*{A quick note on context}
This was recorded and written in Summer 2005 by an angry, single, 24-year old male.

Gary Glitter had just been sent to jail for being a massive paedophile and Michael Jackson was on trial in California for the same thing.

Some of the thoughts, actions and opinions cause me some embarrassment now, five years later, but I retain them in the name of \emph{science} - this is a close, uncensored examination into the mind of a (and, no doubt, almost \emph{every}) single, angry 24-year-old male.

\mainmatter
\chapter{Thursday, May 26: London -- San Francisco}
\section*{London}
Day one started early - 3am for me.  I'd spent the night playing World of Warcraft and so hadn't actually slept much at all - something I thought I could rectify on the coach and aeroplane journey.

I won't make the same mistake again.

There's little to tell for this leg - the advantage of getting a 5.30am coach is that it defies physical laws laid down by God herself - that is, the NX403 is \emph{always} late.  \emph{Always}.  Its as though National Express have discovered some kind of wormhole in Space-Time but are too spaz to use it for good - instead, they use it to slow time down.  The theory goes that, if you slow time down long enough, they can delay the coaches so much that they end up cancelling a service - hence saving money on fuel, drivers, wear-and-tear, that sort of thing.

Clearly, the technicians servicing the wormhole only start work at 8am - so I got to Heathrow on time - early, in fact.  There was a brief moment of worry for a few days when we thought we had to check in at a ridiculously early 8am (for a 11.30am flight), a worry we quickly dispensed by using that most nonsensical of devices, the Online Check-In.

I'd always thought that the point of checking in was to let the airline know that you were at the airport and ready to rock and roll - not so!  Now you can check in using the Internets - from \emph{miles} away, surely opening hilarious new opportunities for crazy capers like checking in to a flight when you're on the wrong continent.  Ha!  That'll screw them up at the departure gate.  The bastards.

Now, I love Heathrow.  I've only been there twice - last year's trip to Amsterdam and now - and it comes second in my All-Time List Of Places I Like To Pretend Is Some Kind Of Starport, ceding first place to Newcastle's Central Station.  Anyway, the thing about Heathrow is that it's \emph{bloody} big - even taxiing to the runway takes about half an hour, so long that I thought we were just going to drive to San Francisco.  Naturally this is all time you have to spend strapped in, roasting 'cause the air conditioning only works while the engines are ramped up - thankfully, Virgin Atlantic's `Attractive Stewardess' policy was in full effect and I got to spend a lot of time leering like the oddball pervert I am.

There were a few good things about the flight, all of which were unfortunately tainted by the single bad thing: since I checked in a little too late to nab the exit-row seats (with the infinite legroom), I foolishly chose the seats behind them, thinking them to be the next-best thing - where in reality, those are probably the worst seats to grab - because, presented with all that legroom, the bastards in front of you will think it's their right to slam their seat-back down all the way for that as-close-to-a-bed-as-you-can-get-it experience.  You'd think not having to sit in the same place with your knees bent for nine or ten hours would be enough for them, \emph{hell no} - the gits want MORE.

If you think I'm being unfairly harsh - after all, the seat backs tilt all that way for me, too, so it's all fair - I say try this:  make fists with your hands.  Press them together and bring them into your chest, so your elbows stick out like wings, or some sort of mutant chicken.  Now imagine you're holding a knife and fork, and you're trying to eat your dinner, but you can't move your hands more than five centimetres from your chest, and \emph{then} you'll see my point.  (A valid retort is `lose weight then, you chunky fuck' - and for that I've no excuse.)

So, the few good things:  First, the enormous amount of food you're given.  Someone at Virgin had the genius idea of keeping everyone eating, all the time, to stifle boredom - so you're constantly bombarded with pretzels, ice cream, sandwiches, tea and coffee and soft drinks, hot food\ldots  All sorts.  It boggles the mind to think that they store so much food on a 'plane, but it was certainly appreciated.  I've already covered the reason my eating experience was somewhat stunted - when your seat-back tray jabs into your flabby extremities 'cause you've absorbed all your allotted room like some sort of amorphous jelly-human hybrid, it kinda takes the fun out of stuffing your floppy-jowled face.

The \emph{second} and best thing is the in-flight entertainment - manifested by a small LCD monitor in the seat in front, and a pleasingly-geeky remote control deck that pops out of your armrest.  Alas, it didn't come with Mario 3 - first-class only I think, disappointing since I'd planned on playing that for eight hours or so - but it DID come with a metric shit-tonne of films and music, on-demand.  I watched \emph{Team America:World Police} and had to really bite my tongue during the incredibly purile (and therefore hilarious to me) bits, like when the hero spends five minutes vomiting, 'cause I didn't want the severe-looking woman sitting next to me know I was wetting my pants at something so childish.

I followed it up with \emph{Collateral}, then stuck the Linguaphone `Learn Spanish' course on, donned the thoughtfully-provided Virgin standard-issue eyepatch things and tried to grab some shut-eye.

I failed.

\section*{San Francisco}
We touched down around 2pm Pacific Time (Pacific Time! How I've longed to live in thee) and soon after was introduced to the Department of Homeland Security, my disdain for which I'm sure needs no discussion.  At the Customs \& Immigration bit we tried to choose the TSA Monkey who we thought had had sex most recently, since they're probably the most likely to not hassle us with stupid and pointless questions like `are you planning on conducting terrorist activities' - the answer to which was already marked on my visa waiver card; on advice of friends and big-ass signs I'd decided against the comedy course of action and ticked the `no, you fucking spackers' box.

Once past America's first line of defense [sic], a slow-witted Mexican, we collected our bags and, starry-eyed and in full-on Tourist Mode, decided to grab my first picture on American soil - the sign saying `Welcome to San Francisco'.  Attracted by the flash, I was \emph{immediately} descended upon by a jobsworth second-line-of-defence, a woman with a serious haircut screaming ``NO PICTUUUURRREEEES!''. Shitting my pants, I immediately complied, only to find later that my battery had failed a half-second before I hit `confirm' and the picture lived on.  It's published with the rest of the pics and I'd like to take the opportunity to say `screw you' to that woman (and that I'm blatantly a raging terrorist, so eat it).

Onwards we go, then, to Dollar Rent-A-Car.  Now, we were promised a Chrysler Sebring, a svelte and sexy specimen of a convertible - what we got, after arguing that no we didn't want a sodding upgrade I don't care how much commission you get, was Herman Munster's hand-me-down - a PT Cruiser.

I'm a guy who appreciates \emph{modern} things, as you may know, so being asked to hang out in this retro beast was anathema to me, and I shed tears.

Thankfully, when you're inside, you almost forget what a monstrosity the thing is, and easily lose yourself in a world of automatic transmission, cruise control and air conditioning.

For some reason, I've become my peers' de-facto navigator, probably because I can read a map pretty well (and I rather enjoy that role) - despite the fact I'm usually crap at it.  Thankfully, whenever we get lost, it's always because my map was (a) rubbish, (b) incomplete, (c) not small-scale enough or (d) all of the above.  Naturally, it's never my fault.

The point is, whoever drew my map o'the day had decided that I didn't need to know which streets were one-way, so my carefully-plotted route to the hotel turned into a `fun' cruise around downtown San Francisco in rush hour, which eventually terminated in what would later turn out to be the \emph{wrong} Holiday Inn - the correct one being just across the road but facing the other way, so we remained ignorant until we realised that the \emph{utter idiot} attending to us had absolutely \emph{no} idea who we were, who Virgin Holidays were, where we could park (sorted out eventually, cost me seventy bucks, many teeth were duly gnashed when we later learned all parking fees were included in the package), et cetera ad nauseum.

Thankfully he minced off eventually, and we had no further need of his so-called `services'.

The remainder of the evening was spent exploring.  We crossed the Golden Gate Bridge, of course, although I'd annoyingly forgotten my camera.  Loath to pay the five buck toll to cross it southbound into the city and, in the first of many instances of what we'd dub the `Scale-Spaz' - the failure to correctly translate distances on a map with \emph{actual} Space - we decided to just carry on exploring, with me foolishly assuring Imran that I could plot a route back along the shore of the Bay and back across the Bay Bridge.

In my defence, we DID have a lovely tour around Sausalito and the southern marches of Marin County (``omg omg Skywalker Ranch is around here somewhere, let's find it''), until the light began to fail along with our own energy reserves and we bit the bullet, my penance being the toll to cross the Bridge again.

Pizza for dinner and back to the hotel - not a moment too soon since we'd both begun to microsleep while driving.

\chapter{Friday, May 27: San Francisco}
An early start thanks to jet lag, and out onto Fisherman's Wharf to find breakfast.  Briefly contemplating the strange hollowed-out bread bowls full of a viscous creamy liquid not too dissimilar to\ldots  Wallpaper paste (wash your mind out - it was clam chowder anyway), we found what we wanted - a retro 50s diner, with free refills on coffee (damn you UK, you need to catch up on THIS one) and that most bizarre of breakfasts, pancakes with massive slabs of lardy butter, syrup and teeny tiny bits of nuked bacon, cheerfully delivered unto me by a friendly Mexican chap who was too young to helpfully advise us on possible drinking venues for that night.

I didn't want pancakes, but it was the only thing for breakfast that wasn't based on eggs.  Breakfast-sans-eggs clearly isn't a popular choice in the colonies.

Thankfully, my map de jour was decent and we found our way to the famous cable cars - where it dawned on me that these were strictly stuck to terra firma, and I stopped looking for the kind that hang from wires on ski-slopes.

Easy mistake to make.

After a full twenty-four hours Stateside, we were terribly disappointed with the lack of hot babes - we'd both swallowed the idea that everyone in the US is beautiful - although it's true that their teeth \emph{are} better than ours.  Anyway, we finally met our first attractive female, trying to get us to subscribe to her charity of choice in the Market district.  I lied and said I already send money to Greenpeace, then took her picture.

All in all, we visited:
\begin{enumerate}
\item A shop full of beauty products - Imran likes to look beautiful; I looked insecure and uncomfortable.
\item The Apple Store, which was a let-down because it was (1) small, (2) dull and (3) not actually all that cheaper, thanks to sales tax - which NO MERCHANT IN AMERICA ADDS ONTO THEIR TICKET-PRICES JUST TO SCREW OVER UNWARY TOURISTS WHO ARE USED TO \emph{SENSIBLE} CONCEPTS LIKE PUTTING THE ACTUAL PRICE YOU ACTUALLY PAY ON PRODUCTS.  The Apple Store was manned by an army of so-trendy-it-hurts fashion victims who I immediately wanted to punch in the face, much like Gap, only with less useful products.

Now, my trusty iPod had \emph{totally} died the night before I left (feel free to send I-told-you-so's to lee@/dev/null), which in a way was handy because I felt less guilty about buying a replacement.  With a 20Gb not being big enough for my library plus future expansion, the iPod Photo being overkill (why on Earth would anyone want to look at Photos on an iPod?) and the Shuffle being waaaaaaaaaaaaaay too small, I decided that maybe I \emph{could} live without having my entire collection attached to me all the time, and spent my hard-earned Benjamins on a 6Gb Mini.  Let me also quickly shoot down the `brand-X-is-better-than-the-iPod' arguments by remind you that I'm hopeless addicted to the iTunes Music Store.

I fondled the Mac Mini for a minute before I suddenly realised that it was, in fact, rubbish - a realisation swiftly followed by the ``actually, this is ALL rubbish!'' thought, which rather neatly quantised the let-down feeling I had - the Apple Store was rubbish because Apple only make, like, four products, and they were all overpriced and fundamentally crap.

I still bought a lanyard for my new 'pod though, but only because I have a thing for lanyards akin to my love of Panama hats. 

\item Old Navy.  I've no idea where on a scale of 0 to Trendy this store lies, but I spent forty pounds on clothes anyway, including a fetching pair of Stars and Stripes boxer shorts which I duly appointed to the office of Lucky Boxers, the incumbent pair retiring citing stress and a desire to spend more time with his family.

\item A pharmacist, where a pleasant Mexican man sold me throat soothing sweets to combat the cold that had been growing since sharing a 'plane with 300 strangers for eleven hours.

Later it would dawn on me that the number of pharmacies in the US easily dwarves the number of schools and hospitals - they do love their drugs.  There were even drive-thru pharmacies\ldots WTF?  You can picture it now: ``hello take your order please?'' ``uuh hi, I'd like\ldots The Codeine combo please?'' ``would you like to go large on that?''  Seriously, get a grip.

\item Skechers.  I bough a pair last year in Amsterdam and they were easily the best pair of trainers I've ever had, and Skechers ousted Nike - who are too busy making shiny green foot-turds these days - as my trainer of choice.  Since the closest stores in the UK are London and Birmingham, I'd been itching to grab a replacement pair for a while, and now I had the perfect opportunity to replace the pair of \textsterling4 Asda trainers I had on, which were \textsterling4 because they contained approximately zero cushioning and were therefore nothing but less-fluffy slippers, not terribly ideal for walking on pavement with.

Again, the servingmonkeys were of the too-trendy variety and seems to find something about me or my accent enormously amusing, occasionally going behind the counter to huddle and whisper to one another, occasionally looking back at me and giggling instead of getting my goddamn trainers from their goddamn stock room.

I paid them back by liberally wafting my foot-scent around their pokey store, the bastards.

\item Macys, where we spent a good deal of time at the sunglasses bit and the toiletries bit at Imran's behest.  Or, rather, Im spent a good deal of time talking to the serving wenches in a language I don't understand, Trendese, while I looked at absurdly expensive smelly water and tried to avoid the aged-yet-still-disturbingly-attractive ladies trying to douse me in the stuff, showing my ignorance by proclaiming loudly `no thanks, I don't wear aftershave' (it's eau du toilette, dahling).

\item Virgin Megastore, where they have a `British' section which sells Heinz Baked Beans.  They don't have beans in the US?
\end{enumerate}

We enjoyed a tasty-ass lunch somewhere in Chinatown and walked back, spamming up and down those big-ass hills - not quite as cinematic when you've got to trudge up and down them repeatedly.

As we ascended the last hill, the noise of rock music drifted down the street towards us.  At first we thought it was some inconsiderate moron with his window wide open, but as we neared its source its true nature became apparent - it was live music, live RAWK music, and it was coming from a \emph{car park}, the car park of Tower Records.  We watched the remainder of the band's set with the large crowd of people gathered, gyrating to the catchy choons - and when it was over we learned that it was none other than Jackie Greene - who we've never heard of.  Enchanted by what we'd just heard, we both entered the building, bought his CD and had it signed by the man himself, who we promised we'd try and make famous in the UK before realising that he was, in fact, a primadonna cock who wears sunglasses indoors, soaking up the praise, while his poor bandmates, skilled musicians down to a man, packed up the gear outside.  To make their day, I stopped them and thanked them.

I'm like that, you see - always rooting for the underdogs.

We wrapped up the afternoon by taking a drive around the city again, this time heading west to check the Pacific out (the Pacific! How I've dreamed of living next to you).  We whacked one of our new Jackie Greene CDs on, only to encounter our first broken promise of the day - rather than the rockin' stuff we'd heard that day, the CD was entirely whiney, soft-ass shite - not even the kind of whiney soft-ass shite that I \emph{like}.  That bastard tricked us!  Boycott Jackie Greene!  Umm\ldots When he eventually decides to take a pop at the British market\ldots

The Pacific coast wasn't at all what I was expecting - it was kind of dull, and cold, but I consoled myself with the thought that San Francisco has it's own microcosmic weather system, and that the rest of the coast was probably all shimmering sand and palm trees, just like I've been led to believe it is.

Naturally we got lost and ended up hurtling through Daly City trying to get back to base camp before the night's fun o-meter reached its peak without us.

Around here is where it started to go a little bit\ldots Wrong.  Heading back downtown we passed \emph{Hooters}, and decided it would be a great yarn to grab some dinner there - to do something \emph{really} `American' - so we did, after showering and donning my latest pair of lucky boxer shorts, fashioned out of a Star-Spangled Banner (``shyeah, if I pull an American bird they'll LOVE this'').  I stopped short of pasting George Bush's stupid fucking face on my arse, though.

It's at Hooters that we felt another promise had been broken, because it wasn't at all what we were expecting.  Sure, there were beautiful, scantily-clad, lithe and full-bosomed serving wenches - but there were also a lot of \emph{families}.  People take their KIDS to Hooters, which totally ruined it for us - no \emph{way} is it possible to get horny when two tables down is some kid's birthday party.

Unless you're an aging, talentless pop star (oooh, betcha didn't see THAT one coming).  There were sad old men watching SPORTS on the TV in there, for God's sake - don't they have a TV at home?

See, for \emph{me} personally the attraction of Hooters is somewhat duller than your average male, simply because I don't really give a shit about women I \emph{know} I have absolutely no chance of ever being with.  Hooters could well have been staffed by clones of Morenna Bacarrin (look her up, Philistine) for all I care, I was there for the FOOD.

Perhaps it was wrong of us to expect good food at Hooters, and even wronger of us to base our opinions of the place on the food rather than the\ldots well, hooters, but it's hard to like a place that serves chicken wings which are 95\% lard and instead of fish, they serve `Cousin of Fish' (I'm not even kidding, the menu said `Cousin of Cod' or somesuch and the waitress even went to far as saying ``well, it's not ACTUALLY cod, it's\ldots'' before I bade her keep her trap shut).

We ate our side-salads instead, and promptly left - I don't care if most of the girls there ARE genius PhD students trying to fund their college courses, they can con their tips out of someone else.

We'd been in town only forty-eight hours and already learned a valuable lesson - ALL the food is made out of lard.  ALL of it.  Even the VEGETABLES have been loving crafted out of butter.  That stereotype about `meals are bigger in the US' isn't true - it just feels that way because you're eating LARD.

We left in a hurry, but not after asking our reasonably attractive waitress where to go out in the city (the question was posed `where would YOU go to', because we wanted to meet young, attractive and, most of all, easy women (you'd have done the same, damn you)) - so after our hasty retreat, we grabbed a cab and ordered the cabbie to take us to \emph{women}.

What we didn't expect was to pick the cab of the most retarded taxi driver in all of San Francisco - I didn't believe it was possible to meet a cab driver who didn't know where young, cool people like ourselves hung out (``what do you mean you don't know where hot women hang out?  Don't you spend all damn night picking them UP from places?'').  Thankfully, after collating our intelligence we picked our way to where the nightlife seems to be centred, and attempted entry into the club our bounteous Hooters girl had recommended.

We didn't actually get very far, because the bouncers turned us away, curtly informing us that we'd have to wait - goodness knows what for, because the place was painfully obviously \emph{dead} - so after shooting them a `well-I-shan't-ever-come-back-here-I'm-British-you-know' glance we instead decided to waste our time in the pub next door.

BIG MISTAKE.

Picture a small, seedy sports bar.  Turn all the lights down reeeeeeally low.  Place six or seven one-armed bandit things and assorted football arcade machines down one wall.  Add two filthy pool tables, and a few ancient CRT televisions spewing out baseball with a hint of purple 'cause the tubes are shot, some more dirt, some nice `generic rock music' - the kind that makes backing tracks on, well, sports videos or Monster Truck shows or something - and \emph{biker dudes}.

I had my doubts; Imran kinda liked it.

After getting some `drinks' and peeling my hands from the sticky bar, we found a table to chill at, recently vacated by some smelly biker rock sports-fan dudes, and instantly realised that I physically could not fit in the gap between the nailed-down table and fixed bench/chair thing.  By then, though, it was too late to turn back and people were starting to peer at us strangely, so I had to suck in my vast, swollen abdomen and just \emph{go for it}, managing eventually to wedge myself in but not before knocking a drink over, and not with the room required to actually \emph{breathe}.

We drank our piss-like beer while I slowly turned blue and kept an eye out for some redneck biker to stab us for being in his regular seat, but it made us ill (probably because it was piss, \emph{actual} piss - it was that kind of place) and we took our leave.

Just to spite the dumbass doormen who denied us entry to our first intended destination, we strolled right past them and onwards towards a club that had actual \emph{women} outside.  But what's this?!  Guest list only?  Not to worry, our man Imran (as ever) has some tricks up his sleeve.  Addressing the guestlist-gorilla in a perfect smarmy minted City-worker accent, he proclaims that he'd heard of this establishment even as far as England, and that it came very highly recommended, and we'd travelled SO far to check it out.

Quickly elevated to Guest List status, we entered in, did a quick reconnaissance sweep, and settled down to get blotto and watch women.

It wasn't long before we were approached by an attractive, pleasant girl, who took a seat beside us and made smalltalk (whereupon I instantly checked out, I hadn't yet reached the correct state of mind to bother with such stuff).  A fine start to the evening, we thought, until it started to become clear exactly who Jamie-Lynn (or Janielyn, or whatever) was - a vulture.  A carrion-bird who worked for the bar, and whose job it was to entertain menfolk and get them to buy inordinately overpriced drinks, before schmoozing off to ply her wares on other groups of y-chromosomes.

We fell for it the first time, eventually giving in to her regular `can I get you boys something to drink?' promptings, and she took off to get us a couple of Buds which, bizarrely enough, don't as much like urine in the US.  Well, \emph{usually} - because once again we'd been bestowed with disgusting swill.  Damn you, Jamie-Lynn.

It took us nine tenths of the bottle, and great deal of protesting from both our stomach linings, to realise that the foolish girl had given us \emph{alcohol-free} beer.  Great!  Now you too can drink excrement and not even get drunk!  What a wholly pointless exercise!  Thankfully she came back (although I suspect only because our bottles were on the wrong side of full, and she felt there was an opportunity to sleaze us into giving her our Benjamins - WRONG!) and we made our disgust very clear.

She got us some REAL beer (well, Bud again - sigh) - and never spoke to us for the rest of the night.  Result.

By this time, the club had begun to fill up a little - with \emph{frat boys}.  They ARE real, and they're every bit as obnoxious as you'd expect a bunch of lightweight college jocks to be, and we wanted them to explode.  Eventually they didn't, and we turned our attentions to a group of young ladies who'd decided to sit close and make the foolish mistake of making eye-contact, clearly our cue to plot an intercept course which, true to form, we did.

Again, I checked out during the initial smalltalk - I wasn't in the mood, and instead had a go at playing the dark, intense, moody guy (the card I'd played most nights out since 1997 (no, it still hasn't worked)) while Im made some headway with an easy-on-the eye lassie who, alas, didn't instantly melt at our accents (which unfortunately was the linchpin of our plan).  While I sipped on my Bud and put on face \#51 - `Sly Grin As Though I'm Laughing At Some Private Joke Because I'm THAT Carefree and Laid Back', I though about something we'd been talking about earlier, the Spock Factor.  We'd deduced that every single-sex group of girls has one Spock in them - the slightly less-attractive girl whose job it was to stop her girlfriends getting too close to icky guys, and generally moan and be a stick-in-the-mud.  You know the people I'm talking about, the ones who'll interrupt your dancing to loudly proclaim that she wanted to go home.  Yes, I'm fully aware of the irony that I'm usually the male-Spock - hey, if I'm not having fun, I'm damn well going to make sure you're not, either.

Back to the story - I'd been figuring out which one of our new friends was the Spock, eventually deciding it was the one without the occupied ringfinger.  My suspicions were confirmed when Im and his quarry got up to get their groove on (I still wasn't in the mood, so I moved to face \#52 - `Cheerful Grin As If I'm Thinking ``you go, guy'' And Pleased That My Friend Is Having Fun With A Girl, Because I'm THAT Carefree And Laid Back And Not At All Jealous') and Spock started looking uncomfortable - eventually dragging her whole crew to the dancefloor before marginalising Im before finally pushing off to the back of the place, wherein lay the frat boys.

Denied.

After a last overpriced beer, a final scout around and finally coming to the conclusion that no new, cool people were entering, I Spock'd Im and we left, seeking more fun and frolics on the streets of San Francisco.  There were some strip joints (`no', I said - `we won't find hot single women there, will we') and some slightly trendy places that were empty, and a few nomadic groups patrolling up and down.  By this stage I'd finally imbued enough to think it's socially acceptable to stop random girls on the streets and scream `WHERE'S THE FUN AT?  WHERE ARE YOU GOING?  CAN WE COME?' - a technique which, as you can imagine, fails to have the desired effect (`ooh, what a kooky, confident English guy, that's kinda cute, let's hang out') and instead, well\ldots chases people off (`oh Christ he's talking to me, let's get the fuck outta here').

There was some action though - a fight.  Just your average drunken brawl in the middle of the road - shit, we even had those in Ponteland - except when the bouncers intervened, it became clear that the bouncers were \emph{packing heat}.  This drove home the point that the country was just as strange and unfamiliar as anywhere else beyond our nice, safe borders - and they speak English to \emph{deceive} us into thinking that we could almost be in England.

We took a taxi home.

Here, my notes turn into a series of exclamations of distaste at the American way-of-life, at the messed-up TV with adverts every five minutes, Fox News and their campaign to turn everyone into xenophobic automata and just how central to some American's lives pharmaceuticals are.  Every third advert is for some sort of wonder-pill.  Every third strip-mall has a drive-thru pharmacy the size of a stadium.  Every problem can be solved with drugs.

Oh, and US TV adverts are allowed to bad-mouth their competition, by name.  It's almost funny to watch some guy saying ``ahh, Biz Automatic can't get rid of THESE stains, because it's shit - but OUR stuff can!''.  Naturally, said guy has to have dubious academic credentials subtitled on, along with his name - \emph{like anyone cares about his name}. ``Hi! I'm Dr. Doug Arseburger [subtitle: Doug Arseburger Ph.D, Leading Cheese Researcher], and our cheese is far better than that crap Cheese, Inc. make'' and so on.

It's also bizarre to see just how low-tech your average dude-on-the-street is.  Walk around somewhere as backwater as Chippenham and even here there are hundreds of those damn white earbuds that USED to say `I'm a pretentious new-media whore' but now say `I'm the same as everyone else' - what we saw in our travels were hundreds of people still with old-skool cassette Walkmans.  It was like the 80s all over again, only less crap.

\chapter{Saturday, May 28: San Francisco -- Yosemite}
\section*{San Francisco}
At this point I'm still a hotel noob, and genuinely believe that if we fail to hand in our cheap plastic keycards back the reception desk before checkout time, \emph{terrible} things will happen, so I spend the former half of the day fussing and worrying.  We breakfast on clam chowder in bowls make of sour-dough and fresh coffee; it was \emph{sublime} if not a little cold and windy on the pier.

The USS Pampanito is moored at the wharf - she's a WW2 submarine that's been converted into a museum, and after enjoying the view of the Bay and Alcatraz, we head inside for a bit of history, briefly stopping to shake my head at a comment book entry by Cindy of Palm Springs, CA that, amongst comments like `it's a privilege to be here', reads simply: `this sux'.  Bless.

We stroll around the shops.  Ever the fashion victim, Imran tries on and discusses the relative merits of a succession of obscenely expensive designer sunglasses, with a decidedly dodgy Russian dealer.  His Oakleys just aren't trendy enough these days, but a pair of Raybans will fill the void nicely - I begin to feel a little dirty, wielding my \textsterling9.99 `no protection, guaranteed' pair from Burtons.

The Russian dude knows his onions, and prompts Im to go in bold new fashionable dimensions (``people think because you have a small face, you have to have small glasses - try these LARGE ones''), which he does before eventually (and amusingly spontaneously considering how long we looked and deliberated) settling on a pair.

The Russian guy rubs his hands, because it's now the REAL selling to start - price negotiation time.  These specs are top of the line and have an accordingly vast price tag, but our man Im is just as shrewd as our Soviet buddy, and eventually a lower price is agreed - but wait!  We have one trick left up our sleeve.

``Aww, I forgot about the sales tax,'' fakes Im.  ``Such a bummer.  I'm not sure I can afford it now.''

With a furtive glance left and right, a faintly discernible screwing up of the face while he eventually decides we're not IRS agents, our comrade lowers his voice and whispers, ``\ldots you pay cash?''

With the faintest of smiles, Imran nods.

``OK.  Pay cash, you pay only ninety dollars\ldots'' There's a pause, and we expect him to finish his sentence with ``tell anyone, and I slit your throat while you sleep''.

It doesn't come to that, thankfully, only I begin to feel some throat-slitting action is imminent as it turns out Im has forgotten his wallet, and the spaz has to run all the way back to the hotel to fish it out - while \emph{I} have to stand around the tiny, empty store looking sheepish.  Russian dude isn't best pleased, and briefly considers murdering me and selling my organs on the black market before again convincing himself that this isn't an IRS sting operation, and we really ARE hapless limeys - and promptly tries to sell me some specs, which I get out of buying by convincingly telling him that I don't ever actually go `outside'.

One successful bit of commerce later, and we jump into the MunsterMobile and hit the road - it's time to do some `\emph{Bullit} shit', by which we mean zoom around the famous streets of 'Frisco.  First stop is one of the highest points in the city, the Coit Tower on Telegraph Hill but there's a mammoth queue.  We give up and decide to snake our way down Lombard Street - apparently the most crooked street in the World, much to the disdain of a traffic cop at the bottom who commands us to the go faster and get on our way, with a look that could easily, even from a vehicle coming at speed down a bendy-ass hill, said `f'ckin' tourists'.

Our final port of call is Twin Peaks - which really IS the highest point for miles, and we soak up some breathtaking views of the Bay and city, spread out beneath us like a Lego playset, from our feet to the watery shore.  While waiting at lights on the road there, we see a pimped-up (chav'd-up?) car pull alongside us and brake to a stop - but what's this?  His hubcaps are spinning away like some kind of crazy optical illusion.  We are in awe of his badass pimpery.

We finally managed to secure my Union Jack to the crossbeam of the car, and once we hit the road for good, I cast it up into the air and it unfurls behind us - a picture of majesty and patriotism, until it gets caught in a vortex, wraps around and threatens to smother Imran while he's driving at speed through a tunnel.  Yeeah - majesty can wait.

I rein the wayward flag back in as we cross the Bay Bridge into Oakland.

\section*{Yosemite National Park}
We pass a good deal of oddballs on the road to Yosemite, some of whom we take pictures of.  We collect images of Puerto-Rican gangsters, annoying children on the back of motorbikes with faces you just want to punch, hard, and a variety of emetic bumper-stickers.

When it's my turn to drive, we pass an SUV full of guys who are clearly on their way to fun and mischief.  There's an instant understanding between our two vehicles, a connection that only a load of guys on road trips can make, and we spend twenty minutes alternately overtaking one another - only when we pass, a guy in the back winds down his window and sticks a cardboard box out.  With a shock we see what looks like a \emph{head} in the box, before we realise that it's in fact HIS head, and he's taped the box to it and his shoulders.

We laugh as the cardboard-clad nutter starts to headbang, his head in actual danger of blowing away - and we ready our cameras and overtake again, grabbing some quality stills.  With a wave goodbye we zoom off, and secretly hope that maybe they were going to Yosemite too.

As we ascend into the Sierras, we ride past some REAL hillbilly stuff - places where the only radio you can receive is a choice of KGOD, KJESUS-103 and my personal favourite, K-YOU'RE ALL GOING TO BURN IN HELL, HEATHEN.  Places where there are more churches than inhabitants, all colours of the Christianity rainbow, and they're all handily listed on massive boards titled 'Places to Worship' as you enter the town.

We wryly note the lack of mosques.

Cedar Lodge is nestled off the road, surrounded by hills and trees, and is very beautiful.  We unpack, grab some dinner at the lodge's really awful 50s-style diner (covered in Elvis stuff - isn't that an anachronism?), chug down some beers and retire.  I watch Wind Talkers on the TV (gritting my teeth as the adverts interrupt every five minutes) and fall asleep to Lostprophets in my new-media-whore earbuds.

\chapter{Sunday, May 29: Yosemite National Park}
In the morning we dine with the Sierra Shadow Casters, who are a group of leather-clad bikers in the traditional Hells Angels style.  While we were in awe of their obvious coolness and secretly wished we could follow in the ways of whiskey-drinkin' and hard-bikin', frankly we were scared they'd kill us and eat our bodies if we happened to be at the breakfast bar at the same time as them, so we waited and made small talk with the five hundred-year old shrew-like Skeletor woman who, with infinite disdain, agreed to boil us some water since, well, we needed it to make \emph{tea}.

It's true what they say about food in the USA, it's all bigger - although I seriously suspected foul play and/or growth hormones this morning as we chomped through strawberries the size of a human head.  Clearly the `fix-it-with-drugs' ethos so prevalent here extends even to pumping strawberries full of\ldots I dunno, strawberryosterone.

The plan was to get out early and, well\ldots I'm lying.  There \emph{was} no plan.  Even the proto-plan - get to the National Park bit before the rush - flopped on its face as we turned out of the lodge and straight into a traffic jam.  Yep - Holiday Season certainly had started.

Our flesh crisped in the sun as we crawled on, following a mighty river flushed with melting snow from the nether regions of the mountains, and into the Park.  Our guidebook suggested we start off by climbing straight up to the highest part - Glacier Point - and soaking up the view before maybe following one of the numerous trails, so climb we did.

At about 6,000ft we began to suspect that `Glacier Point' was quite literally a glacier.  The lush forest and streams quickly gave way to ice and frozen tundra and channels filled with the roaring of liquid pouring from the melting snow heaps.  And it was COLD.   And I panicked because it didn't take long for our fuel gauge to dip into the region I call `oh shit', and soon started suggesting courses of action like putting it in neutral and coasting back down to sea level.

Nevertheless, at eight thousand feet the road ended and we bundled out to meet the most awe-inspiring sight yet - one hundred and fifty degrees of Yosemite National Park (the other sector of the view was blocked by a mountain, duh).

It was enough - there was Half-Dome and, uuh, some more peaks, and some valleys, and - well, they should have sent a poet.  Despite the altitude sickness, the sloshing-ear thing and the mass of holidaying families, I communed with nature and marvelled at the forces that could create such beauty (no, not \$DEITY, \emph{glaciers}).  It was unreal; I wanted to reach out and grab it to make sure it wasn't a painting.

We grabbed some `trail mix' (mmm nuts) and a `powerbar' - basically an emetic slab of frozen peanut butter - before realising that we'd screwed up.  We \emph{should} have parked the car at the bottom of the park, got a free shuttle up and then hiked down.  What HAD to do now was drive back down and miss out, because by the time we got there it'd be too late.

As a matter of fact, thanks to the enormous amount of traffic and slow speeds, we got out of the park and to the nearest (town? village? hamlet?) dot on the map, labelled `El Portal'.  I couldn't have said it better than my erstwhile colleague: ``this is some Chainsaw Massacre shit.''  Thankfully we managed to fill up with gas before being skinned and eaten, and decided to brave the traffic once more in order to check out El Capitan (an impressive monolith which, I'm sad to say, excited me more by virtue of its upstaging of Shatner in Star Trek V: The Final Frontier) and check out some more of the park at ground level.

By the time we found a place to park and start hitting the trails, though, it was getting dark and we were forced to abandon our plan to find and wrestle bears, and plump for a much shorter route, a route which still had us panting and sweating like \$MICHAEL\_JACKSON\_JOKE.

At least until we got to the end, and were rewarded by hanging out on a bridge close to a smaller waterfall, lapping up the spray and absorbing Nature's glory.

In lieu of the long drive we had planned for the next morning, we decided to take it easy that evening and plan the route over a beer back at the Lodge.

While we arranged our checkout, some wise old man sitting at the souvenirs desk overheard our plans and, in a true hillbilly I'm gonna take advantage of your naivety and kill you with a chainsaw moment looked us over and whispered ``you boys heading to Vegas? I got some baaaad news for you''.  Thankfully he wasn't going to cut us up and eat us, but inform us that the Tioga Pass - our route over the Sierras - was \emph{still} knee-deep in snow, and we'd have to find another way.

``The pass is only closed in winter'', I protested.  ``It's almost June!''

``Things don't always go to plan out here'', he rasped.

Our re-navigation had scuppered Imran's desire to forgo the route Virgin had planned for us and burn our way across Death Valley (for which I was secretly thankful, the very idea of straying from the plan scared the crap out of me).  Instead of crossing the mountains, we'd have to follow them south through Fresno and Bakersfield before turning east, skirting around their tip and heading up into Nevada the \emph{long} way around.

It was by all accounts going to be a colossal journey and so we packed up and hit our respective sacks.

\chapter{Monday, May 30: Yosemite -- Las Vegas}
Our road took us through the Californian central valley, the slow decent from the heights of Yosemite taking us past some beautiful countryside including, but not limited to, such alien things as orange groves.  It was my turn to drive; we'd divided up driving duty such that I did the long easy roads and Imran did the urban areas - ostensibly because it had been a few years since I'd owned a car and was therefore a little rusty.  The real reason was, of course, twofold: driving on single-carriageway desert highways disappearing to the vanishing point was my ultimate reason for being, and I was completely paralytic with fear of the idea of driving around the megalopoli we were hurtling towards.

We decided we could spare a few minutes and left our busy highway to duck into a small road lined with orange trees.  It was a road that went to nowhere and we shouldn't have been on it, but the road epitomised everything we were looking for - despite being only half a mile from the busy main route, it was deserted, flat and perfectly straight, the shimmering tarmac rushing on to its date with the horizon.

We absorbed it for a few minutes and rejoined the throng.

Past Fresno we parked at a truck stop for a quick bite, ordering quesadillas from the bored Mexican man-child at the counter.  They oozed with grease as we wolfed them down before filling our tank with gas and resuming our pilgrimage, passing billboard after billboard after bumper sticker reminding us - lest we forget - that \emph{Jesus is Lord}.  They really take that sort of thing seriously, it seems.

At Bakersfield, Imran's desire to abandon the route we had been given and, instead, head into Death Valley finally triumphed over my own, more conservative desire to reach Las Vegas alive.  Besides, he argued, the route we were following took us way too close to Los Angeles, it felt almost like cheating - we didn't want to come close to it until we were actually supposed to be going there.

We turned off the main road and headed into the desert.

\section*{Death Valley}
If we were going to do it, I wanted to do it with a car full of provisions and fuel (you can't be too careful) and so our first stop was California City (or the town of Mojave, since I'm not sure that California City isn't just a crossroads in the middle of nowhere).  Mojave is, of course, now famous for its eponymous Space Port and the home of Burt Rutan's SpaceShipOne - when we arrived it was famous for being dusty, hot and choked with traffic so much so that I lost my temper and, after finally pulling in to the gas station, insisted that Imran should drive.

The gas stations at Mojave kindly provide all sorts of tools and supplies to clean your windscreens of desert dust and dead bugs; we made good use of them after buying many bags of jelly babies.  We would later come to regret our choice of dinner or, at least, our decision to travel without indigestion tablets.

As we turned north-east and headed out into what I've always wanted to believe was my spiritual home, the Mojave Desert, we began to realise that, once again, we'd fallen prey to the Scale Spaz.  As it happens, the distance to Las Vegas through Death Valley is about the same as your average trip to the Moon, and before long the signs of civilisation began to wane.  The trees and orchards began to give way to scrubland and bush and the temperature climbed with every mile we advanced.  Wind farms began to spring up on the hillsides, turbines in huge quantities - and that reminded me of something I'd always wanted to do:  Now that I was out of the driving seat and free to relax, I insisted we play one of my most favourite albums of all time, Kyuss' \emph{Sky Valley}, a record which was made \emph{by} desert creatures \emph{for} desert creatures.  Taking in the big open sky and long, straight, baking hot road we cruised along with nary a road user in sight.  (Of course, anyone worth their salt would know that I was in the wrong area for Kyuss, Palm Desert itself being a great deal further south than us - but at the time I didn't know it, and it didn't matter).

Nature called and we pulled off to a road heading south.  After driving sufficiently far to avoid prying eyes we saw a railway crossing and, slightly before the crossing, a train. I suppose this shouldn't have been a surprise, but this train seemed be modern, fully loaded, hugely long - and utterly abandoned.

It was \emph{eerie}.  It was as if the train was waiting for the crossing to clear (it was) but the driver had just decided to go to pop our for some groceries or something - except of course there's not a living soul (not even a Walmart) for miles around.  We found it fascinating and even mildly scary - something that just shouldn't \emph{be}.  We got as closed as we dared, played on the tracks like children, and took pictures.

Onwards, then, norther and easter, until vegetation failed and we reached China Lake, a dry lake and also the home of the China Lake Naval Weapons Center.  Naturally we were slightly concerned with the idea of being a stone's through away from, I dunno, instant fiery death from a wayward Naval Weapon and so we urged the car forwards with gusto, skirting its eastern border and heading almost due north into yet another dry, salty basin - Searles Lake.  Searles Lake is surrounded by chemical works of some description and really is a perfect candidate for that most peculiar tradition of naming settlements for their primary export (like Borax Springs, CA or Chloride, AZ or Molybdenum, KS (I made the last one up)).  Of course after a few more miles we found out they had done just that - hello, Trona California.

We started to climb, passing beautiful natural landmarks like a discarded bag of fast-food leftovers until we had cleared our hill and saw, spread before us, the Panamint range - the western walls of Death Valley.  We stopped the car and breathed it in - Telescope Peak, the highest point in the National Park, away to our right; the still-blue (but rapidly darkening) sky, the floor itself spectacularly arid and empty and, ahead, our road down into the Panamint valley and up again, heading east to the Park and beyond to Nevada.

Our (well, fine, Imran's) plan was to avoid main roads as much as possible and essentially creep into the valley by the back way.  This we achieved with distinction, only using the main route when it was absolutely necessary and leaving it again at the first opportunity.  With only a (admittedly comprehensive) AA Road Map I'd bought back home, this was tough going and, as dusk fell and the fuel gague went to the wrong side of the quarter mark, I quietly panicked about the possibility of being stranded in Death Valley.  Let's face it, if it was called `Happy Friendly Valley' I might have felt a little better but it was baking hot still; even at eight PM it was \emph{thirty-seven degrees} inside the car - it being a convertible and us being low on gas I vetoed air conditioning.

We climbed the Panamint range and, at its peak, decided to explore what looked like an abandoned missile silo or nuclear bunker.  We fantasised about sitting out Nuclear War here, and that there was probably an entire city underneath our feet but, in the end, we thought it prudent to carry on down into that Purgatorial valley floor.  Past Stovepipe Wells, Beatty Junction and the road snakes out ahead - not another car to be seen, so we put pedal to the metal.  Despite mild panic about our fuel consumption, we feel exhilarated as we power through the hot desert air at speed, past mountainous sand dunes looking wholly out-of-place this side of a beach; top down and music loud.  There is an upturned, abandoned sofa on the side of the road.  We wonder who on Earth travels to Death Valley and throws a sofa away.  Signs point to places ahead of us, Furnace Creek; Badwater (the lowest point on the entire continent).  It is pitch black.

Soon we reach the `tourist hub' of the National Park, Furnace Creek.  It is deserted but there is, finally, a self-service gas station - I could cry with joy, but lose the game of rock, paper, scissors and so I am the one who must fill up the car - surrounded by empty darkness, fearing that `anything could happen out here'.  I fill up, \emph{quickly}, while Imran keeps alert and ready to flee in case there are, well, hillbilly murderers.

We make Furnace Creek eat our dust and fly east, deliberately keeping up the eerie tension by imagining zombies and killers around every turn - until we see a human step out on to the road in front of us.

Imran slams the brakes on and we screech to a halt in front of a man, standing alone, his hand out as if to ward us off.  We look at each other and whisper `fuuuuuuuuucccck!'.  Imran keeps his foot a mere atom's width from the accelerator.  We are ready to fight for our lives.  The man is surprisingly dressed for a murderer, he is wearing a high-visibility reflective jacket and his murder weapon is a long `Stop' sign.  His name is Chris.

Chris tells us that there are some roadworks ahead and we must wait with him for an `escort car'.  He's a clever one.  We don't trust him.  He radios an associate and tells him to 'bring the truck'.  The Zed and Maynard scene from \emph{Pulp Fiction} play through our heads.  We share a nervous glance.  He tries to make smalltalk but we're not in the mood; Imran turns the radio on to cut him off.  We wait for him to strike.  He bides his time.

Eventually a truck arrives.  Chris and the driver have a chat.  The truck turns around and, on its back, there is a sign saying `Follow Me'.  We follow it.  Chris waves goodbye and hopes we enjoy Las Vegas.  We had escaped\ldots but \emph{for how long?}

Death Valley Junction and, somewhere in the darkness, we cross the state line into Nevada.  I lament not being able to get a picture of the sign informing us of this (there may not even have been one) but have to focus on the more important issue of `where the heck are we'.  It is now very late and, having left the heat behind, the chill begins to set in.  We apply more gas and hurtle through the night.

Eventually, up ahead, bright neon lights begin to rise from the horizon.  We grow excited - could this be Sin City, rising to meet us?  There are signs and hustle and bustle galore as we near the city limits, casinos and motels, a welcome change from the day's desolation - but what's this?  ``Welcome to Pahrump, NV''?!  It's not Las Vegas.  It's some horrific copycat city between it and us.  Close-up, it looks downtrodden.  Depressing.  We feel a little sorry for it.

We step on the gas.

\section*{Las Vegas}
Fuel crisis averted, I now begin to panic that for some reason, the hotel won't let us check in because it's too late at night.  We would have to sleep in the car.  On Las Vegas Boulevard.  This won't do.  A line of hills rises up and comes towards us - Mountain Springs.  We climb it, reach the top; the nose of the car tilts down a little - and there it is.  Las Vegas fills the floor of the great basin ahead of us with light.  Innumerable lights and, in the midst of them all, a great spike of light pierces the sky - of course it's the light atop the Luxor, a beacon drawing everyone's gaze to the Strip (which, as any pedant knows, isn't actually in Las Vegas proper).

It is spellbinding.

Route 160, across the Las Vegas Freeway, and onto the Boulevard.  We turn north and head up.  We're staying in Mandalay Bay at the extreme southern edge of the strip itself and I insist we head straight there rather than prowl up and down the strip itself (a decision I'd later regret, everyone should take every opportunity to drive the Strip at night).  We pass the iconic `Welcome to Las Vegas' sign and high-five with glee, before turning left and crossing to the front of the Hotel, standing majestic, golden and tall, immediately south of the Luxor's pyramid and searchlight.

There's a little confusion since the signs outside say `Four Seasons' rather than Mandalay Bay, but in our delirium we ignore that; according to the map and signs, we're in the correct place.  Imran is impressed - apparently Four Seasons is quite posh, perhaps `Mandalay Bay' is simply a franchisee of some sort.  A valet asks if he can take our luggage and we agree, privately agreeing not to tip him.  With infinite bravado we approach the front desk and ask to check in.

We're not on their system.  We have no reservation.

Impossible - we show the clerk our booking vouchers.

The clerk points to the part of them that say ``Mandalay Bay''.  He then points to the sign behind him that says ``Four Seasons''.  We point to the map and road signs.

Ah yes, he says.  The Four Seasons, although physically attached to the Mandalay Bay, is an entirely separate enterprise, and we must pack our stuff up, leave, turn around and go around the `back' way.  We are most certainly in the wrong hotel.

We recover our baggage from the bemused valet, fail to tip him for bringing them ten metres, walk right back out of the door, pack our stuff up, leave, turn around and drive up the strangely thin and winding `back' way.

The valet whisks our car away; it's the first time I've had to deal with this concept and I have to admit I don't really like it - would we have to pay every time we parked?  It seemed like a bit of a pain.

I was soon soothed by the luscious interior of the hotel.  Of course there's really no such thing as being `too late to check in', \emph{especially} in Las Vegas, and within moments we were enjoying the spacious, decadent, air-conditioned lobby and elevator up to our spacious, decadent air-conditioned room.  We were very high up, facing south - away from the city, with a wonderful view over the airport and badlands beyond.  There was an almost comical contrast with the view directly beneath our window, that of a tropical watery wonderland (it's Mandalay Bay, after all) complete with pools, wave machines and sprinklers - a fantastic display of our disregard for our arid surroundings.

We were filthy and tired but there was no way we were going to turn in without first having a look at the casino floor.  After a freshen up and a change of clothes, we headed down to check it out.  Once past the tacky gift shop, there is an ocean of slot machines; beyond the slot machines there are the tables - blackjack and roulette of course, as well as some esoteric ones.  Beyond the tables are the game rooms and next to the game rooms is the crazy Sports Book section, with its wall of video monitors and easy chairs.  The air is fresh and cool, the place is clean and modern - it is our first glimpse of a `real' casino and we're impressed.

Alas, this being our first time, we are way too meek to even attempt any `real' gambling.  We are worried that we might do something wrong and end up in a back alley being violently beaten by large security guards (looking back on it we can admit this is somewhat unrealistic); we don't know where to get chips from - we don't even know if it's \emph{okay} to just wander up to a roulette table and do our thing.  Is there a queue?  Would the other guests frown on us?.  We are also terrified of the minimum bets - five dollars?!  Forget it!

In the end, we decide to pointlessly and fruitlessly waste a few moments on some slot machines before turning in.  I am so exhausted I fall asleep in my clothes and, sick with what we've decided to call `desert fever', we both have a restless and unsatisfying night.

\chapter{Tuesday, May 31: Las Vegas}
We awake feeling incredibly dehydrated.  The industrial-strength air conditioning of the hotel has sucked the moisture from our bodies; it's a small price to pay for such blessed chilly comfort.  We decide to break our fast in one of the hotel restaurants.

I have smoked salmon bagels for the first time and decide they are the best thing since sliced bread.  After filling up we decide to investigate our immediate surroundings - the hotel.  It's \emph{big}.  It takes us half an hour to get from one end to the other and there even seems to be a fully-equipped zoological aquarium or shark exhibit or \emph{something} at the far end.  We have no real interest in such things (and especially since it seems to cost an inordinate amount of cash) and so we return to the casino proper.

We arrive just in time to catch the croupier's shift change and an incredible earful of profanity from one of the early-morning gamblers at a roulette table we were slowly trying to build up enough courage to sit down at.  He's clearly been there all night and he clearly wasn't winning.  He calls the croupier a funking motherhubbard; the croupier washes his hands of us and leaves.  We're enthralled.

Our courage fails and we retreat to some slot machines.  I try to get to grip with that most pervasive of games, Video Poker - I've never played \emph{real} poker before but somehow manage to fluke my way into winning one dollar twenty five cents.  I hold my first winnings voucher close to my heart and vow never to cash it in, indeed, to frame it.

Imran, still suffering from Desert Fever, decides to have an afternoon siesta and retires back to bed.  I decide it's high time I hit the strip.

Leaving Mandalay Bay I turn left - north - and begin my magical journey.  Luxor is on my left, its black edifice shimmering in the sunlight, every bit as magnificent as last night.  After that - Excalibur, all towers and parapets but looking unfortunately underdressed compared to its neighbour: New York, New York - a bewildering construction.  My mind boggles as I try to figure out exactly which parts are hotel rooms that people stay in and which parts are simple fabrications for the cityscape.  The famous roller-coaster trundles up and down, in and out.

I cross the Boulevard's six lanes - groaning with traffic even at midday on a Tuesday - to the eastern edge and continue up. Here the buildings take on a more traditional look; the Tropicana, MGM Grand (and its bonkers roomie, `M \& M World'.  I crossed the road again, taking pictures up and down the strip, every bit the tourist.  I was baking hot and \emph{deeply} regretting my decision to wear black and go on a walkabout at midday \emph{in the desert}.  Eventually I sought refuge underneath the Aladdin (now Planet Hollywood) just short of the utterly ridiculous scale model of the Eiffel Tower in front of `Paris'.

I found myself in the Desert Passage Mall beneath the casino (now `Miracle Mile Shops') - it was cool and dark and its decor was wonderful, faux-Middle Eastern complete with cobblestone paths and even an alarmingly convincing skybox covering the ceiling.  I wandered the streets for a little while before finding myself staring lustfully into a rather expensive looking hat emporium called \emph{Hattitude}.

Now, as has been documented, I have an odd predilection for hats.  On my last trip abroad I'd bought a wonderful white floppy paper fedora from, um, Tie Rack or Accessorise or something; I was very much attached to it and had many months of sterling service from it, but it had mysteriously vanished a few weeks before I left.  Distraught at the idea of being abroad with my head uncovered, I'd bought a white cotton fedora just for this trip but it really hadn't been working out for me; it was uncomfortable and \emph{deeply} uncool.  In front of me, however, in this shop, was a fantastic straw hat of quality and fashion.

I wanted it.  I wanted it, but couldn't be sure it wasn't for, well, women.  Having no fashion sense myself, I rely on others to tell me when I really can't wear something - but this time I'm left with little choice but to ask the shopkeepers, a couple of snooty ladies.  My opening gambit is a straight-to-the-point double-whammy of ``does this make me look cool?  Are you sure this isn't for women?''  Of course, seeing nothing but an easy forty dollars, the ladies all quickly agree that yes, I did look cool and no, it really wasn't a gender-specific item.

I bought it and coveted it.  Made in Mexico. I take this as a mark of high quality and am pleased as punch.

I deem it high time Imran joined me and so I leave the underground warren, don my (sun protection guaranteed) hat and head south back to the hotel.

After a swift change of clothes I repeat the walk but this time with my erstwhile parter in crime in tow.  We marvel at the sights, the sounds, the people - and always the ever-present heat beating down on us.  We find a restaurant in the Desert Passage and eat a wonderful lunch - but then we make a critical mistake.

We didn't tip the waiter enough.

He hadn't really done a very good job and, hell, we wanted to save our cash so that we could pointlessly throw it away gambling later.  It's a rule I've always lived by - tip according to the quality of service received - but (of course) it's not a rule our colonial cousins appreciate and this was made \emph{very} clear to us on the waiter's face and attitude as he uttered various voodoo curses under his breath and vowed to kill our firstborn.

Deeply shamed, we stuff some more notes under our plates and leg it.

We head out of the mall and decide it's time for us to finally start wasting money \emph{properly} by entering the nearest casino and heading to the game floor.  Since I've come to the conclusion that the slot machines are completely pointless and that there is no entertainment or drama to be gotten from them at all, we waltz right past them.  There are poker rooms, the idea of which I find very sexy but, alas, I don't actually know how to play poker, so they're right out.  There are blackjack tables but I'm incredibly worried that I cannot do the simple mathematics required on demand and the idea of the game `pausing' while I frantically try to sum the value of my cards while everyone watches fills me with cold dread.  This leaves me with one alternative - roulette. 

I'm still too chicken to try it, though, and Imran, growing impatient, decides that the best way to look cool is to start on the blackjack table.  He dives in and I stand and watch, shaking with adrenaline.  He looks So Cool.

Roulette looks easy.  I lurk for a good long time, watching punters `play', and I'm pleased when it transpires there's actually no skill involved at all.  There also looks to be an `easy mode', where people not hardcore enough to want to bet on single numbers or any of the baffling array of rows or columns or corners can just whack ten dollars on `red' and cross their fingers.  This is how I start out - red or black, on the lowest minimum-bet table I can find.

Of course, with the one to one odds, the wins come nearly as much as the losses and pretty soon I'm mad with the power and betting exuberant sums like ten dollars.  Imran isn't doing too badly either; we're immediately identified as High Rollers and begin to receive complementary drinks from passing waitresses.  This being our first time, we're astounded at the concept and naive enough to think this makes us special.

Buoyed by our success, we decide to move on - there are a million more casinos to Take Down yet.  We head out and north, further up the strip, underneath the faux Eiffel Tower, plodding on past the Flamingo before crossing again just north of Caesar's Palace.  There's a ridiculously posh shopping center there and we decide to have a poke around while fantasising, as many have done before us, about Winning Big and coming back to buy suits that cost more than our own mothers.  Of course, there's nothing there within our budget and so we exit, our secret hopes of seeing some celebrity or other doing a spot of retail therapy dashed.

It is beginning to get dark and so we decide to head home.  Walking south we have our first taste of the other side of Las Vegas, the seedy porny side, when an old man wan wanders up to us and, through his crooked teeth and in the most comical hick accent, shouts `you boys wanna see some titties?' before trying to force a business card for a strip joint on us. `Come check this place out', he begs, `it's the reeeeeeal deeeeeal!'.  We shuffle off as fast as our Stiff Upper Lips will allow and, loathe to run the gauntlet of hundreds of business card and flyer-toting hobos (and quite exhausted from our travails thus far), we decide to catch a bus back.

We decide that tonight is the night we're gonna go crazy and, hopefully, score with some ladies (and hopefully not the kind that charge by the hour).  Earlier in the day we'd found a spiffy nightclub next to the casino floor of our own hotel and so we elect to check it out.  After a swift bit of gambling on the slot machines where I win a magnificent five dollars, we eat dinner in the hotel's restaurant, head to our room, freshen up and dress to kill.

Looking every bit the pimp daddies we are, we assault the roulette tables in force.  I decide that playing the red/black game just isn't worth it and move up to the heady heights of - gasp - betting on the dozens for a delicious two to one payout.  This works for me but Imran is having no luck - pretty soon he accuses me of being a jinx and moves off to find a table for himself.  After I make 25 bucks and Imran has lost the same, we move on to the bar - Rum Jungle - where we begin to drink.

It is disappointingly sparse but, before long, we're buying drinks for and chatting to a pair of ladies from upstate New York.  As is customary for us, we pretend to be architects and investment bankers in town to spend some of our ridiculously inflated salaries.  They are suitably impressed but the more we listen to them the more we realise that they're actually obnoxious rich girls who can afford their own damn drinks.  They come tantalisingly close to uttering something I've longed to hear since we'd gotten off the 'plane - \emph{I love your accent} - but in the end, they decide that they'd had enough and disappear to make out with some other guys.

We drown our sorrows with flaming sambuka and the bartender takes pity on us by handing out some free drinks.  We appreciated the gesture but not the foul-tasting liquid he'd given us and decide it was just a trick to help him clear his past-its-sell-by-date booze shelf.  After a while it becomes clear that the place not only isn't going to fill up with fine honeys who really \emph{do} melt at the sound of our clipped British tones but isn't actually going to fill up with anyone \emph{at all}.  We decide to stop buying absurdly expensive rounds for each other and, instead, get drinks for \emph{free}.  By gambling.

We stumble around the hotel for a little while looking for a kebab before absorbing ourselves with that more adult of pastimes, intoxicated roulette.  Imran slopes off to see if the mockingly-positioned ATM (\emph{Spent all your money in pointless gambling?  Don't worry!  Just get some more!}) will work with his limey bank card - it does, although of course only after subtracting a substantial percentage for the privilege - and he withdraws a hundred dollars.

He saunters up to my table, swaps it for a single hundred dollar chip (there are many dramatic gasps) and, with infinite swagger, puts it on `black'.  ``What the hell,'' he says, ``it's Vegas.''

It's red 12.

``It's okay, man - I can make it back'' he asserts, before disappearing to the ATM again.  After a moment he returns with yet another shiny new Benjamin Franklin, one hundred dollars.

Chips are exchanged.

Onto black it goes.

Everybody gasps.

It's double-zero.  \emph{Everybody} loses.

``I think I'll go to bed now,'' he whispers, defeated.

I break even.

\chapter{Wednesday, June 1: Las Vegas}
After another night of desert fever exacerbated by booze, loss and despair, we wake up very late.  Today we plan to check out some of the other hotels (resorts? Casinos?  Whatever) along the strip and so we waste no time getting streetside.

We decide to take the bus north but are too lazy to cross the street and so end up getting on the closest - southbound - vehicle.  There's one one stop further south than Mandalay Bay and that's McCarran International Airport; unfortunately for us it was its final stop and so we were unceremoniously disgorged into the terminal to wait 20 minutes for the next one.  We are shocked to discover slot machines lining the walls of the airport; even in this sterile, functional environment it seems people can't get enough of throwing money away.

Finally the northbound bus departs and takes us as far north as we dare go - Wynn's.  I know that, galactically speaking, that's not very far north at all - in fact it's some way south of lots of famous, popular places like the Stratosphere and Fremont Street - but at the time, with our limited knowledge of geography, we believed the Strip \emph{ended} about here.  We knew, of course, that there was a lot more to Vegas than the mile of street we'd been on - we just didn't know \emph{where} and, frankly, we were too afraid to find it.  Instead, we went into Wynn's.

We didn't Wynn.

We lost.

We lost so badly that they stopped serving us free drinks and so we abased ourselves by \emph{pretending} to feed some slot machines coins so that it \emph{looked} like we were gambling.

They were too clever for us, though, and we left.  We'd spent little time on the west side of the Strip and so we crossed over in one of the ever-so-handy raised pedestrian walkways before turning south.

We pass Treasure Island but catch it at the wrong time - there are no feisty pirates to be seen, although we very much appreciate the frontage.  Imran next points out the Bellagio and its famous fountains although, once again, we're at exactly the wrong time to witness their synchronised acrobatics.

It is again baking hot and, as we trudge onwards, we pass many shops selling the most awful tat.  Our interest is piqued by shops that rent out supercars for ridiculous rates; we've seen many of these (filled with rowdy chavs, to a car) up and down the Strip and briefly consider joining in since, of course, the ladies can't resist pasty tourists in powerful automobiles.  Naturally this would only be a secondary concern since our real raison d'\^{e}tre would be to go out into the desert and put pedal to the metal on a Road to Nowhere.  In the end, though, we decide that the heavy traffic around the city would (a) make us look uncool and (b) make getting \emph{out} of the city prohibitively expensive, and we walk on.

Our minds turn to dinner and, somewhere between Aladdin's and the Hard Rock, behind what looks like a bunch of crazy circus tents on the pavement, we enter a Chinese buffet with high hopes.  Unfortunately the food is mediocre at best and we leave unsatisfied.  It's getting late, now, and again the streets are awash with people selling flesh.  Men follow you with sheaves of suspect business cards for establishments of ill repute, loudly and annoyingly rasping through them as if shuffling cards; almost daring you to look at them.  They bring to mind the brazen drug dealers of Amsterdam's Red Light District, swarthy men repeating `coke... coke... coke...' as you pass, only this time it's `titties...'.

We eventually arrive almost at our own doorstep and stop at the Luxor.  It is wonderful inside; looking up at the inside walls of the pyramid we can see the hotel rooms and the bonkers slanted elevators required to get there.  The ground floor is equally mind-boggling.  There are many levels and sub-levels, some with caf\'{e}s, others with box offices for the various shows, still others with fashionable shops.  There are replica Egyptian buildings everywhere and we can't even \emph{find} the casino floor (it is, apparently, underneath us).  It is most confusing and oddly difficult to navigate.

We make it our mission to get into one of the angled elevators and take a trip up to the balconies so that we can get a good look down into the interior and, hopefully, throw some stuff, but we're scuppered when it turns out you need keycards to get anywhere and we're stuck on terra firma.

We take the short and wholly pointless monorail from the hotel back to the Mandalay Bay right next door.  We decide to have one last flutter on the roulette tables.  Imran is still blaming me for his bad luck and shuffles to the table next to mine, steadfastly sticking to only betting on red/black whereas I am sticking on the dozens.  Thankfully we are not stupid enough to fall for the Gambler's Fallacy (what happened before will influence what happens next) or that rubbish `strategy' of betting double every time you lose (sure you might win it all back in one fell swoop, but \emph{what if you lose?}  It doesn't take long for Imran to finally bow out having blown it all; I am up fifty dollars which pretty much means I broke even.  I believe myself to be a Gambling God and we call it quits; it will be a long drive tomorrow and so we turn in early.

\chapter{Thursday, June 2: Las Vegas -- Grand Canyon}
We retrieved our car from the valet early on.  We hadn't seen it the whole time we'd been in Las Vegas and I had all kinds of flashbacks to \emph{Ferris Bueller's Day Off}, where the valet takes the classic car for an epic joyride.  Of course our fugly Chrysler is far from classic and we find it in exactly the condition we left it.

Waving goodbye to the Strip, breakfast comes courtesy of a diner next to the gas station in downtown Las Vegas just before we turn onto the Boulder Highway.  Before long we're queuing to cross the Hoover Dam, a magnificent construction to be sure but we can't quite believe what is essentially a major road road runs right along the top of this valuable landmark.  It feels a little like someone annexing Buckingham Palace's driveway to the M25.

We cross under the humming power cables and snake over the top of the dam between the perfect, shimmering blue lake on our left and the sheer drop on our right.  Coming up towards us on the other side is the state line, demarcated by a large `Welcome to Arizona' flag.  It's the first such sign we've seen, despite Arizona being our third state, so we try to make a fuss over it.  Thankfully there are some parking spaces nearby so we slot in and jump out, cameras and my well-travelled Union flag in tow.

Many passing cars toot as we fly the flag beneath the sign and we cheer, taking photographs that I would later realise feature our magnificent emblem back-to-front.  I imagine there are Department of Homeland Security laser sights on me as we do this, the red dot creeping up to my forehead, some jobsworth just \emph{itching} to take down these ballsy terrorists.  Time passes and I am not killed, so we head out.

After Kingman on Interstate 40 the road straightens out into a long straight highway.  It's certainly sparsely populated but it's not the archetypal Long Straight American Desert Road; this is a quite dull, modern motorway with absolutely no opportunities to pull off at random and make some dust trails.  I am driving and, after a while, I decide the ridiculously conservative speed limit of 75 is just too damn slow.  I disable the cruise control (it is anathema to me in any case) and open the throttle.

Right past a parked, concealed Highway Patrol car.

The car immediately pulls out and begins following me.  After a while he even comes up level with me on the inside lane.

``You should slow down,'' suggests my co-pilot.

``No,'' I said.  ``If I do that, it's like admitting I was doing something bad.  If I just carry on like nothing is wrong everything will be fine,'' I reasoned.

``That's fucking retarded,'' said Imran.

This is obviously a sentiment shared by the police officer.  Despite keeping my eyes fixed straight ahead and not looking anywhere near the car immediately off my left elbow, despite the fact that I know through some crazy E.S.P. that he is pointing a radar gun at my \emph{head}, I carry on as normal.  It doesn't take long for the flashing lights to start and it's only really then that I begin to think I made a mistake.

I panic a little.  In the reams of literature so helpfully provided by our holiday company, none of it discusses how to deal with being pulled over by the cops.  I briefly entertain the idea of opening with a clich\'{e} like ``what seems to be the trouble, officer?'' but decide against it since, well, we all know exactly what the trouble is.  Instead, I sit quietly in the drivers' seat and wind the passenger window down.

``Sir, the reason I've stopped you is, in the state of Arizona, it is not permitted to exceed the speed limit.''

I toy with the idea of asking if there is a state in the Union where it \emph{is} permitted.

The policeman seems a little glum; it's obvious he's looked up our registration and seen it's a hire car.  He asks for my licence and positively crumbles when he hears my (admittedly mid-Atlantic) accent - we're clearly hapless tourists and his usual dose of brutality and degradation may well cause a bit of a political stir.  I feel like a colossal tit as I rummage around to find my licence, naturally it's buried at the very bottom of my suitcase which itself is buried underneath Imrans'.

``What I'm gonna do today, sir, is give you a written warning'' he concedes, holding my Communist State-issued driving licence with obvious disgust.  I feel a small flush of pride as he scribbles `Ponteland, UK' down as my address; the name will now live forever in the vaults of the Arizona State Highway Patrol files.  ``Drive carefully and obey the law, or next time you'll get a fine,'' he warns.  I fantasise about shouting ``DIPLOMATIC IMMUNITY!'' at him but decide not to push it.

In the end I'm given the written warning.  It was for driving at eighty miles-per-hour on a seventy-five miles-per-hour deserted, desert motorway.  Good job, Officer.  We drive off, and I set the cruise control for 75.

It's a wonderful day for driving, perfect weather and clear blue skies.  I'm breaking my hat in and we're eating up the miles, admiring the surroundings.  At Williams we turn north for another hundred miles and eventually our next port of call - the Grand Hotel - comes shooting towards us out of nowhere, all alone on the side of the highway.  It takes me by surprise and I literally burn rubber as I slam on the anchors.  We'd been travelling so long it actually felt strange to be stationary.

Although we have only a limited time at the Grand Canyon, it's late and we require food.  We discuss the possibility of a helicopter tour but it's just too expensive for me and, besides, the unfortunate fact is we'd have to leave quite early tomorrow and retrace our steps back to Kingman and beyond.  As we dine in the restaurant, the hotel lays on some entertainment for the punters - it's some Native Americans, singing and dancing.  Imran and I feel a little embarrassed - the fate of the Native American people makes me sad at the best of times - and we joke about how they're probably singing about how the White Man killed his culture, land and, well, People - all the while fat white tourists like myself clap with glee at their antics.

There's a pool at the hotel and Imran decides to take a dip - I decide to simply relax on a lounger reading my book (Iain Banks' \emph{Consider Phlebas}) since the pool doesn't look like it's been cleaned since the Colorado first dug its channel into the rock a mile north of us.  Indeed the are floating clouds of copper discolouration which we deeply hope is just bromine.

Afterwards, we sip on some beers and reminisce about old times.  We recall all the childish fun we had as students and also lament how, disregarding our current adventure, Real Life since graduating from University is actually a bit rubbish.

\chapter{Friday, June 3: Grand Canyon -- Lake Havasu}
Immediately after breakfast we head for the `Grand Canyon IMAX Experience'.  I've seen neither an IMAX film or the Grand Canyon and I'm looking forwards to a treat.  We decide that, since we can't really spend enough time around here, watching the film would be the best thing to do since it could cover far more than we could hope to. It certainly was a treat but of course it merely whetted our appetites for more.  We parked up on the South Rim and took a walk to the edge.

For me, the thing which immediately stood out was just how flat the horizon is.  It's almost a mockery of the jagged creeks below; it's almost like you could miss the fact there's an incredible, deadly gorge beneath you.

And then you look down.

Obviously it's beyond my means to describe, simply because my mind cannot make sense of it.  There are giant pinnacles and sheer cliff faces, odd cathedral-like structures that you can just imagine are secret bases for cultist rejects from \emph{Temple of Doom} - some of them even have what look like entrance doors.  And it goes on and \emph{on}.  And it's \emph{really} deep.

While chasing a bizarre looking insect to photograph, we strike up a conversation with a travelling group of Ultra-Catholics.  One of them is a girl who appears to be wearing a jumper adorned with Scripture and pink fluffy slippers.  We give her a high-five for style and she \emph{finally} comments on how cool our accents are.  Mission Accomplished.

Time has passed and so we race south, back down the road towards Williams and westwards again towards Kingman.  Thankfully there are no cops and we're free to soak up the sunshine.  As usual, our map would have us follow the Interstate for most of the journey and, as usual, Imran disagreed.  He suggests we `wing it'.  I look at my road map which, rather handily, lists sites of interest and tourist spots and I'm finally convinced to agree by two things -- first, a site marked `ghost town' and second, a site marked `Black Mesa'.

Now, as any good \emph{Half-Life} fanboy knows, the Black Mesa Research Facility which features as the setting for the first game is actually in New Mexico - but I wasn't about to let that fact ruin the thrill of visiting a completely different fictional location and so I acquiesce.  We turn off I-40 just after Kingman and onto a single-carriageway barely paved road leading, seemingly to nowhere.

I decide I love Arizona right there and then.

\section*{Kingman and Oatman}

Before long we get as close as we can to `Black Mesa' and park up; it's apparently still miles away and we can't really see anything black or mesa-like.  What we \emph{can} see, though, is a large butte sticking out of some scrubland a way off the road.  We both need to answer nature's call and so decide to wander up to this huge clump of rocks and, well, wee on it.

En route we dodge strange cacti - we're fascinated, having never seen one up close before.  We are surrounded by desolation punctuated by electricity pylons and, on the horizon, we can see hills (and indeed, something that looks like a dark mesa...) and nothing else.

As we climb the butte, we worry about snakes - but mostly we worry about \emph{Tremors'} Graboids - a fear that only grows as we depart and head down what according to my map is an alternate route to Lake Havasu.  We follow the road due west and before long it turns into dust.  At regular intervals we cross roads laid perpendicular to us which are nothing more than ditches - sometimes not even that.  They lead nowhere that we can see; our road seems to be taking us straight into a hill - we have faith that there will be a pass up and over the hill at the end.  The roads crossing us have, at first, fairly romantic names - Tombstone Trail, Desert View Trail, Teddy Roosevelt Road.  When they run out we cross Agate Road, Flint Road, Garnet, Zircon and Jade Roads.  Finally we cross roads named for Native American tribes - Havasupai, Hopi and so forth - presumably named after the people who were here before being replaced by a series of isolated caravans.  People obviously live in these trailers (indeed I later learn this is a township called `Golden Valley', which sounds both romantic and like a filthy Karma Sutra manoeuvre) and we get the ominous feeling that we really don't belong here.

We put pedal to metal and hurtle west, the hill ahead of us looming ominously.  Massive plumes of dust billow out behind us - in the clear air and flat surroundings they can be seen for miles.  Any minute now the trail will turn right - south - and lead us home.  Some people are coming out of their trailers to watch us.  We worry about hillbillies and again, more than ever, Graboids.  Of course, any film connoisseur will inform you that \emph{Tremors'} Perfection is in Nevada but, again, shut up.

Then we run out of road.

No southbound turn.

No route over the hills.

Just a sign, ridiculously and superfluously pointing out the `dead end'.

There's nothing for us to do now but drive back through our own dust clouds back to the interstate, which we begin to do.  We pass a dilapidated caravan in a fenced-off lot; along the side of it is scrawled in red paint `for sale' and a telephone number.  We fantasise a little about buying it (it certainly looks cheap enough).  I'm actually started to warm to Golden Valley - there are magnificent views of the Black Mountains to the west, Hualapai Mountains to the east and glorious, flat empty desert plains north and south.  Some day I may choose to live there, but right now, the clock is ticking and we're miles from home.

We reach the butte again and realise that there is another road running almost parallel to the one we've just had the abortive experience with, and it looks just large enough to be the one on my map.  We follow it and, sure enough, we cut through the desert and, finally, begin to rise up over the mountain ahead.

Nestled high up we run into a gas station - just as well since, once again, my blood pressure is heightened by our dwindling fuel supplies.  Unfortunately for us, it looks like it has been abandoned since 1950.  It makes up for it, though, by being in that classic 1930's style that brings Route 66 to mind - it's the first time we've seen anything like it.  Second, just left of it is a huge saguaro cactus, standing proud and defiant.  There is nothing more `Arizonan' in my mind, and I fall in love a little more.  Driving on, we're surprised to see that despite our blunders and for all our `winging it' attitude, we are actually travelling on Historic Route Route 66 itself - there are respectful, reverent signs informing us so.  It feels like fate.

As we crest the hill, we see a beautiful expanse of terrain beneath us, sliding away to the horizon.  Dusty green plains stretch dramatically out and away, ruffled and undulating like a thick quilt, punctuated by small hills and yet more buttes.  It's truly breathtaking.  The road is incredibly winding and we drive like we're a rally team, zigzagging down to meet the soft, springy carpet of the land beneath us - we both spontaneously cheer and whoop with the sheer pleasure of it all.

After a while we pass through Oatman, once a dying mining town that has since reinvented itself as a tourist spot - there are saloons old-timey restaurants featuring places to tie up your horses.  It was the last thing we expected to see and driving through it feels incredibly surreal.  Crucially, though, there is no gas station and so once again I petition Imran to continue our descent in neutral.  He laughs me off.

The sun begins to sink toward the horizon and the air begins to chill as we plough south through the glorious countryside.  At this stage our map was no help at all; we had no idea where we were and no idea how far our destination was until, finally, salvation - we hit a main route complete with helpful signposts and, thank goodness, a gas station.  We fill her up and, pointed without a doubt in the correct direction, zoom towards Lake Havasu City while watching the sun set behind the hills on the western horizon.  On the way we pass a varied array of huge pick-up trucks, shiny 4x4s, all manner of powerful vehicles and all of them towing trailers on which are secured a cornucopia of beautiful powerboats and pleasure yachts.

\section*{Lake Havasu City}
When we finally reach the civilisation, it's very dark and very late.  Our maps and route plans do not tell us how to get to our hotel so we're navigating using the Force; it's only really been sheer fluke we've done so `well' so far - but tonight our luck runs out and we cruise up and down through the mean streets.  Our only clue is its name - Nautical Inn - so we \emph{assume} it is somewhere close to the lake itself.

As it happens it couldn't \emph{be} closer to the lake - it's on a bonkers island, not far out off the shore but enough to throw us off the scent.  Eventually we arrive and unpack the car to the noise of wild partying - there are cool hip young people wandering about and we see a lively bar built on pontoons bobbing about off the hotel's private beach thing.  All thoughts of travel-weariness disappear in a flash and we silently agree that this is the place we need to be at.  I throw my trusty Union flag over the balcony of our room to let everyone know there are cool, well-travelled eligible bachelors in the hizzouse (no-one actually cares).  We freshen up and throw on our best clothes.  Imran loans me some of his ridiculously expensive - but \emph{oh so worth it} hair sculpting wax and I reverently shape my coiffure before we head on out.

We enter the \emph{Naked Turtle} Bar and can immediately smell \emph{money}.  It's clear the people around us are minted and, to a person, \emph{beautiful}.  Men, women, young and old - they all sport firm, muscular bodies and perfect teeth.  We do our best to fit in and start to gulp down cocktails.

Before long we strike up a conversation with a couple of people at the bar and exchange gossip - there's a kid there from Orange County (of which my only knowledge was `posh suburb of Los Angeles' and `they made a TV drama about it') whose dad had just bought a new place in Lake Havasu City; he was in town to try out his new boat.  We douse the flames of envy with our booze and decide not to employ our usual fa\c{c}ade of being plastic surgeons give that we were, in all likelyhood, surrounded by either
\begin{enumerate}
\item plastic surgeons,
\item spawn of plastic surgeons,
\item clients of plastic surgeons or
\item all of the above.
\end{enumerate}

We're introduced to a guy about our age called Bob.  Bob looks cool; he has a Lynyrd Skynyrd t-shirt, a labret piercing, an imperial beard and a shaven head.  He insists we order some tequila with him and come on over to party with his buddies.  We readily agree but we don't know the first thing about tequila.  ``Don't worry,'', he says.  ``Just get the best - Jos\'{e} Cuervo.  This shit \emph{fucks you up!}'' \\
We order Jos\'{e} Cuervo and follow Bob.

% Ryan/Britta Dwaine/Jackie Dan/Lisa
% girl, white T
% dude, red hat (?)
% dude, black tee spiky DAN!
% girl, black top (lisa)
% dude, shirt (ryan)
% britta

% guy, white hati stubble/tash?? (dwaine!) 
He takes us to a group of cool-looking cats - three guys, three girls - and introduces us, before proceeding to get horrifically drunk.  Later on our new buddies will admit they have no real clue who Bob is; he is `just some random dude'.

We get on very well; there is a mellow dude in a backwards hat and many piercings who I'll call Ozzie; his rambunctious lady who I'll call Harriet; a guy in a smart shirt who is built like an American football player called Ryan; a dark, spiky haired fellow called Dan and his dark, elegant lady Lisa; and Britta, a high-spirited Sarah Chalke lookalike.  We gossip and relate our tales of the Road, they listen attentively and talk about their lives.  Bob steals my camera and takes pictures of himself flipping The Bird.

I'm filled with Dutch courage at this point and notice that Britta doesn't have a man to guard her.  I move in for the kill.  We have a round of drinks together; we laugh and take pictures of each other --- well, I imagine she would had she a camera, mostly it was just me taking the pictures --- and she offers to show me her ludicrously expensive ring.  As it happens, it's her birthday and she has just been given a Tiffany ring as a gift by someone quite close.

It's then that I realise I've spent the last hour chatting up Ryan's girlfriend, \emph{right in front of his face}.

Not really knowing who or what Tiffany's is, I pass comment that it really doesn't look all that expensive.  Ryan's face visibly sags.  Britta does not talk to me again.

Soon after we're joined by Dwaine, who is tanned and sports very manly stubble.  He tells us that they are all spending tomorrow on Dan's powerboat and suggests we join them on the lake.  Dan agrees and We nod, vigorously, and we give him our room number.  He tells us to expect a call in the morning and we, fall-over drunk, retire.

\chapter{Saturday, June 4:  Lake Havasu -- Palm Springs}
\chapter{Sunday, June 5:  Palm Springs}
\chapter{Monday, June 6:  Palm Springs -- Los Angeles}
\chapter{Tuesday, June 7:  Los Angeles}
\chapter{Wednesday, June 8: Los Angeles}
\chapter{Thursday, June 9: Los Angeles -- London}
\end{document}
